%\documentclass[]{beamer}
%\usepackage[norsk]{babel}
%\usepackage[utf8]{inputenc}
%\usepackage[T1]{fontenc}
%\usepackage{minted, enumerate}
%\usepackage[colorlinks = true,
%linkcolor = blue,
%urlcolor = blue]{hyperref}
\documentclass[]{beamer}
\usepackage[size=a3, orientation=portrait, scale=1.1]{beamerposter}
\usepackage[norsk]{babel}
\usepackage[T1]{fontenc}
\usepackage[utf8]{inputenc}
\usepackage{graphicx,epstopdf,tikz}
\usepackage{gauss,amsmath}
\usepackage{siunitx, enumerate}
\usepackage{lipsum}
\usepackage{animate}
\usetheme[slogan=norsk,style=horizontal, mathfont=serif]{NTNU}

\definecolor{NTNUblue}{HTML}{00509e}
\definecolor{NTNUlightblue}{HTML}{6096d0}
\definecolor{NTNUorange}{HTML}{ef8114}
\definecolor{NTNUbrown}{HTML}{cfb887}
\definecolor{NTNUred}{HTML}{b01b81}
\definecolor{NTNUgrey}{HTML}{bebebe}
\definecolor{NTNUcyan}{HTML}{3cbfbe}
\definecolor{NTNUviolet}{HTML}{482776}

\usepackage[style=apa, backend=biber]{biblatex}
\addbibresource{bibtex.bib}

\begin{document}
\begin{frame}[t]
   \begin{center}
      \huge \textbf{\color{NTNUorange}{Shader-Kunst} \color{NTNUgrey}og programmering som verktøy for matematisk læring}\color{black}
      \vspace{1cm}
      \Large \color{NTNUred} \\Beklager \textbf{meget} kladdaktig poster  - ferdig versjon sendes før samlingen\color{black}
      \noindent\makebox[\linewidth]{\rule{\paperwidth}{0.4pt}}
   \end{center}

   \begin{columns}
      
      \begin{column}{0.33\textwidth}
         \begin{block}{Matematikk og programmering i LK20}
            I LK20 fikk programmering og «algoritmisk tankegang» mye større plass.
            Spesielt matematikkfaget fikk inn nye læringsmål rundt numeriske metoder og programmering.
            Også på ingeniørstudiene ved NTNU kommer programmering inn i matematikkfagene, og det er blant kollegaer blitt interessant å
            spørre seg på hvilken måte programmering kan \textit{hjelpe til} på matematikkforståelsen. Vi har blant annet
            en forskningsgruppe sammen med høyskolen i Volda (\href{https://www.hivolda.no/forsking-og-utvikling-0/forsking-ved-hvo/forskingsgrupper/pedagogikk-og-didaktikk/programmering-matematikkforstaelse}{lenke her}), og
            et Erasmus+ prosjekt på gang hvor man undersøker hvordan programmering kan styrke matematikkforståelsen.
         \end{block}
         \begin{block}{Shader kunst og estetikk}
            På ingeniøremnene handler det å ta programmering inn i matematikken, ofte om å lære numeriske metoder og bruke
            programmering til å løse faglig relevante beregningsproblemer.

            Jeg tror det kan være interessant å undersøke om \textit{shader kunst} kan brukes som verktøy for å undervise både matematikk og programmering.

            En \textit{shader} er et program som kjøres på grafikkmaskinvare, og skiller seg fra et vanlig dataprogram ved at det kjøres parallelt. Det kan feks kjøres samtidig for allet pikslene i et vindu, for alle trekanter i et 3D-objekt.
            Shaderkunst handler om å skrive slike shadere å lage grafikkunst. Se feks linken i figuren under.
            %I stor grad undervises samme matematikk til ingeniører fra flere studieprogram, slik at det ikke nødvendigvis er trivielt å gi numeriske problemstillinger som er relevant for alle studenter som tar emnet.
            
         \end{block}
         \begin{figure}[ht]
            \centering
            \animategraphics[autoplay, loop, width=\textwidth]{30}{images/shadertoyframes/shadertoy}{1}{113}
            \caption{Shadertoy eksempel, se animasjon \href{https://www.shadertoy.com/}{shadertoy.com}}
            \label{shadertoy}
         \end{figure}
      \end{column}
      
      \begin{column}{0.33\textwidth}
         \begin{block}{Muligheter}
            Når man skal lage noe slikt som shadertoy-eksempel, behøves en del programmeringsferdigheter, men også matematisk forståelse.
            Alt man vil vise må man beskrive matematisk eller algoritmisk, og man kan ta utgangspunkt i dette 
            for å undersøke lineær algebra, funksjoner, differensiallikninger, fraktaler og sikkert voldsomt mye mer.
            
            \color{NTNUgrey}\lipsum[1]bla bla \autocite{gascoigne2014tacit}
         \end{block}
         \begin{block}{Begrensninger}
            Det er plagsomt mye kode i rammeverket lenge før vist noe på skjermen, og koden kan avhenge mye av hvilken platform man bruker (windows, mac, linux, opengl, direct3d osv).
            Det gjør at programmeringsbiten fort kan komme i veien for matematikken, som er det jeg vil komme til å undervise.
            
            \color{NTNUgrey}\lipsum[1]
            \lipsum[1]bla bla \autocite{gascoigne2014tacit}
         \end{block}
      \end{column}
      
      \begin{column}{0.33\textwidth}
         \begin{block}{Problemstillinger}
            Jeg ønsker å undersøke hvilken betydning det har at man velger å skape noe estetisk som shader-kunst, i stedet for å angripe et faglig relevant men særs konkret beregningsproblem.
            Hvilke fordeler og ulemper har det?

            Videre ønsker jeg å lage et undervisningsopplegg i matematiske metoder 3 for dataingeniører hvor vi undersøker differnsiallikninger, og løsningsmetoder for de, ved å lage litt shader-kunst.
            
            5. semester dataingeniørstudenter tåler en del kode i rammeverket rundt og i selve shader-kodingen. Jeg ønsker også å se hvilke muligheter det er for å bruke
            dette i matematikkundervisning for de som ikke går et ingeniørstudie innen data. Finnes det muligheter for å kode shaderkunst og undersøke matematikk uten å dykke dypt i rammeverket rundt?
         \end{block}
         \begin{block}{Utviklingsarbeid}
            
            Se forrige blokk
            \color{NTNUgrey}
            \lipsum[2]
         \end{block}

         \begin{block}{Bibliografi}
            \printbibliography[heading=none]
         \end{block}
      \end{column}
      
   \end{columns}
\end{frame}

\end{document}
